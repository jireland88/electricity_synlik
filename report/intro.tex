\section{Introduction}

This project uses `Synthetic Likelihoods' \citep{wood_2010} to perform parameter inference for electricity spot price models. The regime-switching model proposed in \cite{huisman_mahieu_2003} will be used to model the spot price. Additionally, this project will evaluate the use of the Robust Covariance Matrix \citep{huber_1967} as an estimate of the sample covariance calculated during the evaluation of a Synthetic Likelihood. Computational benefits will be explored and the resulting fits compared.

Synthetic Likelihoods were first introduced as a method of statistical inference for chaotic ecologial dynamical systems \citep{wood_2010}. In these systems, slight changes to model parameters have large effects on the resulting trajectory \citep{may_1976}. This makes the probability density function unsuitable for determining statistical fit. Thus, traditional likelihood approaches such as Maximum Likelihood Estimation do not work \citep{wood_2010}. In \cite{wood_2010}, this is solved by a) choosing a vector of summary statistics that characterise the underlying dynamics of the model and b) maximising the likelihood of the observed data presenting these statistics.

Spot electricity prices present frequent extreme observations or price spikes \citep[p.~21]{pilipovic_2007}. These spikes are subject to the `double penalty' effect \citep{haben_2014}. This means that sample trajectories produced even from the same parameters can look completely different. Thus, traditional likelihood-based inference is impossible. This project will show how Synthetic Likelihoods can elegantly overcome this obstacle.

The Robust Covariance Matrix was proposed as an estimate of the asymptotic covariance of the Maximum Likelihood Estimator when the underlying model is misspecified \citep{huber_1967}. However, it will be shown that the Robust Covariance Matrix can also be used within the Synthetic Likelihood framework to provide significant computational benefit.

The structure of this project is as follows. In Section~\ref{sec:model}, a model for spot electricity prices will be explored. Then, in Section~\ref{sec:sl}, Synthetic Likelihoods will be introduced alongside the Robust Covariance Matrix. In Section~\ref{sec:fitting}, the electricity spot price model will be fitted using Synthetic Likelihoods. Finally, in Section~\ref{sec:eval}, the goals and results of the project will be discussed.