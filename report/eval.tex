\section{Conclusion}
\label{sec:eval}

This project set out to

\begin{enumerate}
    \item use `Synthetic Likelihoods' \citep{wood_2010} to perform statistical inference for electricity spot prices; and
    \item evaluate the use of the Robust Covariance Matrix \citep{huber_1967} as an estimate of the sample covariance calculated during the evaluation of a Synthetic Likelihood.
\end{enumerate}

On the first point, this project can conclude that Synthetic Likelihoods can be used to fit electricity price data. This is backed up from the simulation study results and the trajectories sampled using the Nordpool data. Further, the methods provided can be used under desirable modifications to Model $\mathcal{A}$, such as replacing the Normally distributed spikes with Log-Normal spikes. This modification would not be possible under the Kalman Filter methadology employed in \cite{huisman_mahieu_2003}.

Further, this project has found Synthetic Likelihoods to be a powerful method of statistical inference. The ability to use a collection of heuristics for parameter inference is particularly beneficial. The difficulty is choosing the statistics. This project demonstrated the use of correlation diagrams to quickly obtain an estimate of the quality of a set of statistics. The statistic $\hat{p}$ was proposed, which successfully captured the frequency of the spikes generated by Model $\mathcal{A}$.

On the second point, this project found the Robust Covariance Matrix to provide significant computational benefit. In particular, a $37 \times$ reduction in the time to fit Model $\mathcal{B}$ when computing the Hessian analytically. Thus, computing the Hessian by hand is a tedious but computationally desirable step. Additionally, the restriction to $M$-Estimators is significant and this project was not able to fit Model $\mathcal{A}$ using only $M$-Estimators.

Whilst both initial aims of the project were fulfilled, there is scope for further work. First, in finding a collection of $M$-Estimators that can fit Model $\mathcal{A}$ and second in extending Model $\mathcal{A}$ to include Log-Normal price spikes. Additionally, more work fitting the Nordpool data would also be beneficial. Summer/Winter seasonality and fundamental changes caused by, for example, the rising gas prices in 2021/2022 present challenges in this area that future projects could look into.

In summary, both aims of the project were satisfied: Synthetic Likelihoods were found to be effective at fitting models for electricity spot prices and the Robust Covariance Matrix provides significant computational benefit to Synthetic Likelihood Estimation. Further, this project provides fertile ground for future projects to explore, particularly in extending Model $\mathcal{A}$ and fitting the Nordpool data.

The code used to produce all figures is available in the GitHub repository: \url{https://github.com/jireland88/electricity_synlik}.